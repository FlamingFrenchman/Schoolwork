\documentclass{article}

% If you're new to LaTeX, here's some short tutorials:
% https://www.overleaf.com/learn/latex/Learn_LaTeX_in_30_minutes
% https://en.wikibooks.org/wiki/LaTeX/Basics

% Formatting
\usepackage[utf8]{inputenc}
\usepackage[margin=1in]{geometry}
\usepackage[titletoc,title]{appendix}

% Math
% https://www.overleaf.com/learn/latex/Mathematical_expressions
% https://en.wikibooks.org/wiki/LaTeX/Mathematics
\usepackage{amsmath,amsfonts,amssymb,mathtools}

% Images
% https://www.overleaf.com/learn/latex/Inserting_Images
% https://en.wikibooks.org/wiki/LaTeX/Floats,_Figures_and_Captions
\usepackage{graphicx,float}

% Tables
% https://www.overleaf.com/learn/latex/Tables
% https://en.wikibooks.org/wiki/LaTeX/Tables

% Algorithms
% https://www.overleaf.com/learn/latex/algorithms
% https://en.wikibooks.org/wiki/LaTeX/Algorithms
\usepackage[ruled,vlined]{algorithm2e}
\usepackage{algorithmic}

% Code syntax highlighting
% https://www.overleaf.com/learn/latex/Code_Highlighting_with_minted
\usepackage{minted}
\usemintedstyle{borland}

% References
% https://www.overleaf.com/learn/latex/Bibliography_management_in_LaTeX
% https://en.wikibooks.org/wiki/LaTeX/Bibliography_Management
\usepackage{biblatex}
\addbibresource{references.bib}

% Title content
\title{AM214 Homework 1, Numerical Part}
\author{Robert Bergeron}
\date{October 19, 2020}

\begin{document}

\maketitle

% Problem 1
\section{}
\[ \frac{dN}{dt} = r_0N(1- \frac{N}{K}) \]
First, we shall convert \(N\) to \(\hat{N}\).

\begin{align*}
    \frac{d\hat{N}}{dt} = \frac{1}{K}\frac{dN}{dt} \text{, or } \frac{dN}{dt} & = K\frac{d\hat{N}}{dt}
    & \quad \text{Start by taking the derivative of \(\hat{N}\) with respect to \(t\)} \\
    K\frac{d\hat{N}}{dt} & = r_0K\hat{N}(1 - \hat{N})
    & \quad \text{Perform the substitution} \\
    \quad \frac{d\hat{N}}{dt} & = r_0\hat{N}(1 - \hat{N}) \\
    \frac{d\hat{t}}{dt} & = r_0
    & \quad \text{Then, we do the same thing for \(\hat{t}\)} \\
    \frac{d\hat{N}}{dt} & = \frac{d\hat{N}}{d\hat{t}}\frac{d\hat{t}}{dt} = \frac{d\hat{N}}{d\hat{t}}r_0 \\
    \frac{d\hat{N}}{d\hat{t}}r_0 & = r_0\hat{N}(1 - \hat{N}) \\
    \frac{d\hat{N}}{d\hat{t}} & = \hat{N}(1 - \hat{N}) \\
\end{align*}

The equation we are left with is equivalent to the logistic equation where both
the carrying capacity, \(K\) and the growth rate, \(r_0\), are one.

% Problem 2
\section{}
This function has fixed points at \(0\) and \(1\), which can be seen easily
by looking at \(\frac{d\hat{N}}{d\hat{t}}\). If we set
\(\hat{n}(\hat{t}) = \frac{d\hat{N}}{d\hat{t}}\) and take the derivative of this
function, we find that \(\hat{n}'(\hat{t}) = -2\hat{N} + 1\). We evaluate at
the fixed points, getting \(\hat{n}'(0) = 1\) and \(\hat{n}'(1) = -1\). \(0\)
is then an unstable fixed point, while \(1\) is stable.

The flow diagram would depict the slope approaching zero the closer \(\hat{N}\)
came to the line \(\hat{N} = 1\), and approaching \(-\infty\). The closer the
initiat condition is to one, the smoother the curve becomes. Like the logistic
equation, the function the flow diagram characterizes would be sigmoid, or s-shaped,
given that the function demonstrates the Allee effect.

\pagebreak

% Problem 3
\section{}
The numerical results pretty much reflect what I got out of my analysis. Though I
should probably have noted that solutions would only be sigmoid if their inital
condition was between zero and one. Otherwise, they just asymptotically approach
one (in case the initial condition was above one), or approach infinity (in case
the initial condition was below zero).

% Problem 4
\section{}
\(\frac{d\hat{N}}{d\hat{t}}\) is a Bernoulli equation, so we solve it by substitution.
\begin{align*}
    \frac{d\hat{N}}{d\hat{t}} - \hat{N} & = -\hat{N}^2 \\
    \hat{N}^{-2}\frac{d\hat{N}}{d\hat{t}} - \hat{N}^{-1} & = -1 & \quad \text{ let } v = \hat{N}^{-1}, \frac{dv}{d\hat{t}} = -\hat{N}^{-2}\frac{d\hat{N}}{d\hat{t}} \\
    -\frac{dv}{d\hat{t}} - v & = -1 \\
    \frac{dv}{d\hat{t}} + v & = 1 & \quad p(t) = 1, \; \mu(t) = e^{\int d\hat{t}} = e^{t} \\
    \int \lparen e^{t}\frac{dv}{d\hat{t}} + e^{t}v \rparen & = \int e^{t} \\
    ve^{t} & = e^{t} + c \\
    v & = \frac{e^{t} + c}{e^{t}} \\
    \frac{1}{\hat{N}} & = \frac{e^{t} + c}{e^{t}} & \text{reverse the substitution} \\
    \hat{N} & = \frac{e^t}{e^t + c} & \text{do a little algebra} \\
\end{align*}
Plugging in the arbitrary inital condition \(\hat{N}(0) = \hat{N}_0\), we find that
\(\hat{N}_0 = \frac{1}{1 + c}\), which, with some work, give us that
\(c = \frac{1 - \hat{N}_0}{\hat{N}_0}\). We replace \(c\) by this and finally arrive
at the analytical solution for arbitrary \(\hat{N}_0\):
\[ \hat{N}(t) = \frac{e^{\hat{t}}}{e^{\hat{t}} + \frac{1 - \hat{N}_0}{\hat{N}_0}}
              = \frac{\hat{N}_0e^{\hat{t}}}{\hat{N}_0(e^{\hat{t}} - 1) + 1} \]

\end{document}
