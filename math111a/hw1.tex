\documentclass{article}

% If you're new to LaTeX, here's some short tutorials:
% https://www.overleaf.com/learn/latex/Learn_LaTeX_in_30_minutes
% https://en.wikibooks.org/wiki/LaTeX/Basics

% Formatting
\usepackage[utf8]{inputenc}
\usepackage[margin=1in]{geometry}
\usepackage[titletoc,title]{appendix}

% Math
% https://www.overleaf.com/learn/latex/Mathematical_expressions
% https://en.wikibooks.org/wiki/LaTeX/Mathematics
\usepackage{amsmath,amsfonts,amssymb,mathtools}

% Images
% https://www.overleaf.com/learn/latex/Inserting_Images
% https://en.wikibooks.org/wiki/LaTeX/Floats,_Figures_and_Captions
\usepackage{graphicx,float}

% Tables
% https://www.overleaf.com/learn/latex/Tables
% https://en.wikibooks.org/wiki/LaTeX/Tables

% Algorithms
% https://www.overleaf.com/learn/latex/algorithms
% https://en.wikibooks.org/wiki/LaTeX/Algorithms
\usepackage[ruled,vlined]{algorithm2e}
\usepackage{algorithmic}

% Code syntax highlighting
% https://www.overleaf.com/learn/latex/Code_Highlighting_with_minted
\usepackage{minted}
\usemintedstyle{borland}

% References
% https://www.overleaf.com/learn/latex/Bibliography_management_in_LaTeX
% https://en.wikibooks.org/wiki/LaTeX/Bibliography_Management
\usepackage{biblatex}
\addbibresource{references.bib}

% Title content
\title{MATH 111a Homework 1}
\author{Robert Bergeron}
\date{October 16, 2020}

\begin{document}

\maketitle

% Introduction and Overview
\section*{Page 35}

% Problem 3
\subsection*{Problem 3}
    For arbitrary \(a,b \in G\), we have
    \begin{align*}
        (ab)^2 & = a^2b^2 \\
        (ab) (ab) & = (aa) (bb)
    \end{align*}
    The order of the operands is different, yet the values are equal, so the elements
    of \(G\) commute and it is therefore abelian.

% Problem 6
\subsection*{Problem 6}
    Let \(S_3 = \{e, \psi, \psi^2, \phi, \phi*\psi, \psi*\phi\}\), \(x = \psi\),
    \(y = \phi\). Then,
    \begin{align*}
        (\psi*\phi)^2 & = \psi*(\phi*\psi)*\phi \\
                      & = \psi*(\psi^{-1}*\phi)*\phi \\
                      & = (\psi*\psi^{-1})*\phi^2 \\
                      & = e*\phi^2 \\
                      & = \phi^2 \\
                      & = e \\
                      \text{On the other hand, } \\
        \psi^2*\phi^2 & = \psi^2*e \\
                      & = \psi^2 \\
    \end{align*}

% Problem 10
\subsection*{Problem 10}
    Suppose \(a,b \in G, a \ne b\), show \(ab = ba\)
    given that all elements in \(G\) are their own inverse. \\

    By the property of this group, we have \(a = a^{-1}\), \(b = b^{-1}\), and
    \(ab = (ab)^{-1}\). Expanding out the latter, we get \(ab = b^{-1}a^{-1} = ba\),
    showing that the elements of \(G\) commute and therefore that G is abelian.

% Problem 11
    \subsection*{Problem 11}
        G is a group, so each element has an inverse. There are \(2n, n \in \mathbb{Z}^+\)
        elements in \(G\), and \(e\) is its own inverse, leaving us with \(2n-1\)
        elements whose inverses are unknown. The inverse of an arbitrary element \(a\),
        which we shall refer to as \(a^{-1}\), must be unique. If we were to attempt to
        pair off the elements of \(G\) in order to find an inverse, we would be unable
        to do so as there are an odd number, excluding \(e\). There must then be one
        element, \(b\), left over, which is its own inverse, hence \(b^2 = e\).

% Problem 12
    \subsection*{Problem 12}
        G is closed under an associative product, which satisfies the first two
        prerequisites for being a group. All that remains is to prove that inverses
        and the identity element exist under the given conditions.

        \begin{enumerate}
            \item Identity: by (a), \(ee = e\), and therefore under (b), \(y(e) = e\). \\
                \(ae = a \rightarrow ae = ee^{-1}a \rightarrow ae = eea \rightarrow
                  ae = ea = a\) and \(e\) is the identity element.
            \item Inverses: Since the element \(y(a)\) for arbitrary \(a \in G\) described
                in (b) is also a member of \(G\), it follows that (b) likewise applies
                to \(y(a)\), such that \(\exists \; y(y(a))\) where \(y(a) * y(y(a)) = e\).\\
                From there:
                \begin{align*}
                    y(a) * y(y(a)) = e \\
                    a * y(a) * y(y(a)) = a * e \\
                    (a * y(a) * y(y(a)) = a \\
                    y(y(a)) = a \\
                    y(a) * a = a \\
                \end{align*}

                And \(y(a)\) is a true inverse, and \(G\) is a group.
        \end{enumerate}

\section*{Page 46}

% Problem 1
    \subsection*{Problem 1}
        \(H \cap K\) contains at least one element, \(e\).
        Since \(H \cap K\) is at the very least a nonempty subset of \(G\), by lemma 2.4.1,
        we need only prove that:
        \begin{enumerate}
            \item \(a,b \in H \cap K \rightarrow ab \in H \cap K\): the former statement
                implies \(a \in H,K\) and \(b \in H,K\) and by virtue of being subgroups,
                \(ab \in H,K\), which implies \(ab \in H \cap K\).
            \item \(a \in H \cap K \rightarrow a^{-1} \in H \cap K\): the former
                statement requires that \(a \in H,K\), and by virtue of being subgroups,
                \(a^{-1} \in H,K\), which implies \(a^{-1} \in H \cap K\).
        \end{enumerate}

% Problem 4
    \subsection*{Problem 4}
    \subsubsection*{a)}
        We perform this proof in the same way as problem 1:
        \begin{enumerate}
            \item Let \(ah_1a^{-1},ah_2a^{-1} \in aHa^{-1}\). Take
                \((ah_1a^{-1})*(ah_2a^{-1}) = ah_1(aa^{-1})h_2a^{-1} = ah_1h_2a^{-1}\).
                Since \(h_1h_2 \in H\), it follows that \(ah_1h_2a^{-1} \in aHa^{-1}\).
            \item Let \(aha^{-1} \in H\). It's inverse is \((aha^{-1})^{-1}) =
                ah^{-1}a^{-1}\), which is in \(aHa^{-1}\), since \(h^{-1} \in H\).
        \end{enumerate}
        Therefore, \(aHa^{-1}\) is a subgroup of \(G\).
    \subsubsection*{b)}
        Since \(aHa^{-1}\) is \(H\) multiplied by \(a\) on the left and \(a^{-1}\) on
        the right, \(o(aHa^{-1}) = o(H)\).

% Problem 6
    \subsection*{Problem 6}
            \[G = \{e, a, a^2, \dots, a^9\} \]
    \subsubsection*{a)}
            \[H = \{e, a^2, a^4, a^6, a^8\} \]
        \begin{enumerate}
            \item \(He = \{e, a^2, a^4, a^6, a^8\}\)
            \item \(Ha = \{a, a^3, a^5, a^7, a^9\}\)
        \end{enumerate}
    \subsubsection*{b)}
            \[H = \{e, a^5\}\]
        \begin{enumerate}
            \item \(He = \{e, a^5\}\)
            \item \(Ha = \{a, a^6\}\)
            \item \(Ha = \{a^2, a^7\}\)
            \item \(Ha = \{a^3, a^8\}\)
            \item \(Ha = \{a^4, a^9\}\)
        \end{enumerate}
    \subsubsection*{c)}
        Did not complete.

% Problem 13
    \subsection*{Problem 13}
        \begin{enumerate}
            \item Let \(x_1,x_2 \in N(a)\).
                \begin{align*}
                    x_1a & = ax_1 \\
                    x_2x_1a & = x_2ax_1 \\
                    x_2x_1a & = ax_2x_1 \text{ Since } x_2a = ax_2 \\
                    (x_2x_1)a & = a(x_2x_1) \\
                \end{align*}
                So \(x_2x_1 \in N(a)\)
            \item Let \(x \in N(a)\).
                \begin{align*}
                    a = & \; axx^{-1} = xax^{-1} &
                        \text{ Trivially, and by the given property} \\
                    a = & \; xx^{-1}a            & \text{ Trivially } \\
                    xax^{-1} = & \; xx^{-1}a     & \text{ Since both equal \(a\) }\\
                    x^{-1} * xax^{-1} = & \; x^{-1} * xx^{-1}a \\
                    ax^{-1} = & \; x^{-1}a \\
                \end{align*}
                So \(x^{-1}\) has the same property as \(x\) and is therefore also in
                \(N(a)\).
        \end{enumerate}
        Therefore, \(N(a)\) is a subgroup of \(G\).
\end{document}

% Problem 14
    \subsection*{Problem 14}
    Did not complete.
